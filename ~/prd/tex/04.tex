\begin{itemize}
\tightlist
\item
  Poisson Permutation Test to Locate Transition from Extensive to
  Intensive Development
\end{itemize}

Growth in the number of scholars and the number of published scholarly
works is attended by a qualitative transitions between extensive and
intensive patterns of citation. When disciplines are very young scholars
are almost always exploring new or at least unclaimed terrain with
little interest in covering the same ground twice. As disciplines
develop a transition invariably occurs; scholars become much more likely
to retrace familiar ground. Much of the work of this study aims to
understand the significance of this fact.

First the fact must be established. Intellectual terrain is often
imagined as a space of meaning or a population meaningfully
distinguishable ideas. Because ideas cannot be directly observed,
several indicators of their presence have been used, citations chief
among them. Empirically, then, we will start on the basis of the
citations as a useful indicator of ideas. Later we will discuss the
limitations of the ideational theory, and we will present an alternative
interpretation of a citation space. Luckily the facts at issue will not
change.

So, then, it will be useful to treat the terrain of scholarship as the
accumulating stock of already cited references. The act of exploration,
so to speak, of this space consists in the inclusion, in the reference
list of a scholarly publication, of a particular set of citations and
not another. A footprint in this space, left by one publication, may be
represented as a count of each citation pair in the list of references.
This operationalization allows footprints to overlap completely,
partially, or not at all. By enumerating citation pairs or co-citations
instead of their individual counts, we also claim that the meaning of a
reference may vary in combination with other references.

A more empiricist and less theory laden interpretation is to claim that
we may identify how disparate acts of cultural production hang together,
without knowing why they do so. Citations provide merely one kind of
thread, but were we to trace out several more modes of relatedness then
we might provide a fuller picture of the sociocultural structure
underpinning scholarship. Such a task is beyond scope for the present
study, but we can at least specify ignorance ({\textbf{???}}). Clearly
there is much more to the content of a publication than its list of
references. But even considering this narrow slice of its meaningful
content, we are already at pains to generalize from the observation of a
citation pattern to the cause of that pattern appearing in a particular
time and place. It will be difficult, for instance, to posit a choice
mechanism, for we cannot discern whether the inclusion of a reference
was the choice of the author, the editor of a journal, the reviewers
refereeing the manuscript, a colleague listed in the acknowledgements,
an uncredited inspiration, etc. I therefore make no effort to identify
an actor responsible for an included reference, but rather consider it
the outcome of the local art world surrounding the production of that
piece of scholarship {[}c.f. theories of authorship @ ;@ {]}. What is a
critical problem to solve for the intellectual historian may a fool's
errand for the population researcher. It is a mistake to treat any
particular citation, and especially to treat the entire reference list,
as reflective of the choice of the author. Indeed this mystifies the
production process behind scholarship.

An extensive pattern of citation then is one that both introduces never
before cited references and one that favors those extant references that
have been cited the least by others. A Poisson distribution is a simple
first approximation of a random search in this space, and observed
citation counts with a mean below the random pattern (underdispersion)
can be considered to represent the extensive pattern, while means above
the same (overdispersion) may represent the intensive pattern.

This extensive pattern of development may be compared to the
paradigmatic model described by Kuhn ({\textbf{???}}10). Once a
paradigm, in the sense of a model to be extended, takes hold among a
community of scholars, normal science ensues as a process of narrowing
the range of possibilities opened by the paradigm. The specifically
scientific pattern of development is to retrace familiar problems until
they are solved, and then to relegate the solution to one or another
form of black box, such as mathematical codification, textbook
explication, or codification in technology. Familiar ground is in one
moment intensely retraced, and in the next systematically forgotten.
Indeed Kuhn aims to demonstrate that the ideology of cumulative
development in the sciences is a consequence of black boxing, which
serves to render subsequent generations of scientists ignorant of a
history better described by a cyclical or sinusoidal trend.

While pre-history of disciplines are beyond Kuhn's scope, here they are
paramount. This emphasis is based in a hunch that the mechanisms that
govern the genesis of disciplines may be implicated in their ongoing
development.

\textbf{References}
